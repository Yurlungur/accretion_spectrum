\documentclass[]{article}
% Author: Jonah Miller (jonah.maxwell.miller@gmail.com)
% Time-stamp: <2013-12-08 22:19:24 (jonah)>

% packages
\usepackage{fullpage}
\usepackage{amsmath}
\usepackage{amssymb}
\usepackage{latexsym}
\usepackage{graphicx}
\usepackage{mathrsfs}
\usepackage{verbatim}
\usepackage{braket}
\usepackage{listings}
\usepackage{pdfpages}
\usepackage{listings}
\usepackage{color}
\usepackage{hyperref}

% for the picture environment
\setlength{\unitlength}{1cm}

% float controls
% \usepackage{float}
% \floatstyle{boxed}
% \restylefloat{figure}

%preamble
\title{An Exploration of the Temperature of an Accretion Disk Around a
  Pseudo-Newtonian Black Hole}

\author{Jonah Miller\\
  \textit{jonah.maxwell.miller@gmail.com}} 

\date{Fundamentals of
  Astrophysics\\ Niayesh Afshordi\\ Fall 2013}

% Macros
\newcommand{\R}{\mathbb{R}} % Real number line
\newcommand{\N}{\mathbb{N}} % Integers
\newcommand{\eval}{\biggr\rvert} %evaluated at
\newcommand{\myvec}[1]{\vec{#1}} % vectors for me
% total derivatives 
\newcommand{\diff}[2]{\frac{d #1}{d #2}} 
\newcommand{\dd}[1]{\frac{d}{d #1}}
% partial derivatives
\newcommand{\pd}[2]{\frac{\partial #1}{\partial #2}} 
\newcommand{\pdd}[1]{\frac{\partial}{\partial #1}} 
% derivatives with respect to R.
\newcommand{\dR}[1]{\frac{d #1}{dR}}
\newcommand{\ddR}{\frac{d}{dR}}
% comoving derivatives
\newcommand{\Dt}[1]{\frac{D #1}{Dt}}
\newcommand{\DDt}{\frac{D}{Dt}}
% A 3 vector.
\newcommand{\threevector}[3]{\left[\begin{matrix} #1\\ #2\\ #3 \end{matrix}\right]}

\begin{document}

\maketitle

\begin{abstract}
  We attempt to reproduce the work of Shafee et al. in their 2008
  paper \cite{Shafee08} to measure the temperature of a thin accretion
  disk. We consider a thin accretion disk around a pseudo-Newtonian
  stellar black hole. By numerically solving the non-relativistic
  hydrodynamics equations consistently, and assuming blackbody
  radiation from the surface, we find the spectrum of the disk.
\end{abstract}

\section{Introduction}
\label{sec:intro}

Einstein's theory of theory of general relativity predicts the
formation of singularities known as black holes \cite{Wald}. The
gravitational pull of a black hole is so strong that, once an object
passes within a certain distance, known as the Schwarzschild radius,
the object is inexorably pulled into the curvature singularity at the
center of the black hole, never to be seen again \cite{Wald}. One
might hope that these black holes are mathematical artifacts that one
wouldn't really observe in nature. However, the singularity theorems
of Hawking and Ellis demonstrate that, given reasonable assumptions on
the makeup of matter in the universe, black holes are plentiful
\cite{HawkingEllis}.

Indeed, a number of astrophysical black holes have been observed, and
their properties measured
\cite{McClintock06,Liu08,Gou09,Frank}.\footnote{This is only a small
  fraction of the literature on black hole observations.} And although
this work will focus on astrophysical black holes, it is believed that
every galaxy has a super-massive black hole at its center
\cite{PadmanabhanTA}. A black hole's enormous gravity pulls many
objects inwards towards the black hole and these form a disk of
accreting matter \cite{Frank,Melia}. It is this ``accretion disk''
(and the ultra-relativistic jets emerging from the poles of the black
hole) that is most visible from Earth \cite{Frank,Melia}.

To learn about an astrophysical black hole, we study the properties of
the accretion disk and how they are related to the black hole. To do
this, we need a reliable, predictive model that relates the black hole
parameters\footnote{A black hole only has three possible parameters;
  mass, charge, and spin. The No-Hair Theorem guarantees that every
  black hole is completely defined by these three quantities
  \cite{MisnerThorneWheeler}.} to the properties of the accretion
disk. The fluid dynamics of accretion disks are extremely
complicated. In full generality, one needs to simulate a highly
nonlinear system of hyperbolic partial differential equations in (3+1)
dimensions \cite{LehnerReview01, Ott08,Font08,
  BucciantiniZanna13}.\footnote{There are many many more papers to
  cite here. This is only a sampling.} However, this is
computationally expensive and one can often extract useful data with a
much simpler model.

In their 2008 paper, Shafee and collaborators propose one such model
\cite{Shafee08}, which is the continuation of a long history of models
\cite{PBK81,MuchotrzebPaczynski82,Kato88a,Abramowicz88,PophamNarayan91,NarayanPopham93,ChenTaam93,NarayanYi94,Narayan97,Chen97,AfshordiPaczynski03}. The
model uses fluid dynamics and a number of approximations to generate a
nonlinear system of ordinary differential equations (ODEs), which must be
solved numerically. In an effort to understand this model, and to
generate an accessible discussion of accretion disk physics, we
present the following attempt to reproduce Shafee et al.'s
work. Ultimately, we aren't successful in reproducing the numerical
results. However, we effectively reproduce the analytic system that
must be solved numerically, and we believe our numerical failure is
interesting in its own right.

In section \ref{sec:background}, we introduce some key concepts from
accretion disk physics that we need to understand Shafee et al.'s
model. This includes Keplerian motion, fluid dynamics, and the
$\alpha$ prescription of Shakura and Sunyaev. In section
\ref{sec:the:model}, we present a derivation of the boundary-value
problem that Shafee et al. solve numerically. We discuss the
approximation of a black hole using a Pseudo-Newtonian potential. We
discuss how the equations of mass, energy, and momentum conservation
result in an ODE system that we want to solve. And we discuss how to
generate boundary data. In section \ref{sec:numerical:approach}, we
discuss our approach to numerically solving Shafee et al.'s ODE
system, including methods and tests of our implementation. In section
\ref{sec:results} we discuss the results of our numerical simulations
and what approach might work better. Finally, in section
\ref{sec:conclusion}, we offer some concluding remarks.

\section{Background}
\label{sec:background}

In this section, we will discuss some typical features of accretion
disk models, including Keplerian motion, the fluid dynamics required
to develop models of accretion disks and the so-called $\alpha$
prescription.

\subsection{Keplerian Motion}
\label{subsec:keplerian:motion}
In the context of accretion disks, Keplerian motion means circular
motion. Given an arbitrary spherically symmetric potential $\Phi$, we
derive the angular velocity of a circular orbit, also called the
``Keplerian angular velocity,'' here.

Suppose we have a test particle in our potential $\Phi$. Then the
acceleration due to gravity on the particle is
\begin{equation}
  \label{eq:acceleration:due:to:gravity}
  g = -\myvec{\nabla}\Phi \hat{R},
\end{equation}
where $\hat{R}$ is the radial direction. However, if the orbit is
circular, we know that the centripetal acceleration must be
\begin{equation}
  \label{eq:centripetalacceleration}
  a_c = -R \Omega_K^2,
\end{equation}
where $R$ is the radius and $\Omega_K$ is the Keplerian angular
velocity. If we set these quantities equal to each other, we find that
\begin{equation}
  \label{eq:keplerian:angular:velocity}
  \Omega_K = \sqrt{\frac{-\myvec{\nabla}\Phi}{R}}\hat{z},
\end{equation}
where $\hat{z}$ is perpendicular to the orbital plane.

\subsection{Fluid Dynamics}
\label{subsec:fluid:dynamics}

We model accretion disks as fluids orbiting a central compact object,
so we must also discus fluid dynamics. In this discussion, we borrow
mostly from \cite{Melia} and \cite{Thompson}. First, let us define the
following functions of the radius $R$:
\begin{eqnarray}
  \label{eq:variable:definitions:1}
  \rho &\text{ is the density of the fluid}\\
  \label{eq:variable:definitions:2}
  \myvec{v}&\text{ is the velocity field of the fluid}\\
    \label{eq:variable:definitions:3}
  v_R &\text{ is the radial velocity}\\
    \label{eq:variable:definitions:4}
  P &\text{ is the total pressure in the fluid}\\
    \label{eq:variable:definitions:5}
  c_s &\text{ is the speed of sound}\\
    \label{eq:variable:definitions:6}
  \Omega &\text{ is the angular velocity}\\
    \label{eq:variable:definitions:7}
  \Omega_K &\text{ is the angular velocity for a text particle in a Keplerian orbit}\\
    \label{eq:variable:definitions:8}
  H &\text{ is one half the vertical thickness of the disk.}\\
    \label{eq:variable:definitions:9}
  \dot{M} &\text{ is the accretion rate of the black hole.}
\end{eqnarray}

The equations of fluid dynamics encode conservation laws. Therefore,
we want to discuss the change in time of energy, mass, and
momentum. One important issue in formalizing these equations is that
we want to discuss the change due to the internal dynamics of the
fluid and due to the motion of the fluid as a whole. This leads us to
the comoving derivative\footnote{Also called the Lagrangian
  derivative} \cite{Thompson},
\begin{equation}
  \label{eq:comoving:derivative}
  \DDt = \pdd{t} + \myvec{v}\cdot\myvec{\nabla}.
\end{equation}
Now, encoded in the language of fluid dynamics, the conservation of
mass law, also called the continuity equation, is written as
\cite{Melia,Thompson}
\begin{equation}
  \label{eq:continuity:1}
  \Dt{\rho} + \rho\myvec{\nabla}\cdot\myvec{v} = 0.
\end{equation}
The equation of conservation of momentum can be written as
\begin{equation}
  \label{eq:momentum:conservation:1}
  \rho \Dt{\myvec{v}} = -\myvec{\nabla} P + \rho \Omega^2 \myvec{r} + \rho \myvec{\nabla}\cdot\sigma,
\end{equation}
where $\myvec{r}$ is the standard position vector and $\sigma$ is the
stress tensor \cite{Melia,Thompson,WikiNavierStokes}. In this form,
equation \eqref{eq:momentum:conservation:1} is just a restatement of
Newton's second law. Later, we will see that, just as Newton's second
law can be broken into linear and angular parts, this equation can be
broken into two pieces, a linear momentum conservation law, and an
angular momentum conservation law.

Energy conservation is a little different. We actually have two
coupled energy conservation laws: one for mechanical energy and one
for thermal energy \cite{Melia,Thompson}. The mechanical energy
equation is \cite{Thompson}
\begin{equation}
  \label{eq:mechanical:energy}
  \DDt\left(\frac{1}{2}\myvec{v}^2\right) = - \frac{1}{\rho}\myvec{v}\cdot\myvec{\nabla} P + \myvec{v}\cdot\myvec{f}.
\end{equation}
This just tells us that the rate of change of kinetic energy per unit
volume is the same as the work done per unit volume on the fluid. The
thermal energy equation is written as
\begin{equation}
  \label{eq:thermal:energy:conservation}
  \Dt{U} = \frac{P}{\rho^2} \Dt{\rho} + q^+ - q^- - \frac{1}{\rho}\myvec{\nabla}\cdot (k\myvec{\nabla}T),
\end{equation}
where $U$ is the internal energy per unit volume, $q^+$ and $q^-$ are
the rates of heating and cooling in the fluid, most often generated by
viscosity and radiation respectively, $k$ is the thermal conductivity
of the fluid, and $T$ is the internal temperature. Therefore, the last
term is the heat flux out of a unit volume of fluid \cite{Thompson,WikiFluids}. We can also write equation \eqref{eq:thermal:energy:conservation} in
terms of the entropy per unit mass $s$ \cite{Shafee08}
\begin{equation}
  \label{eq:entropy:conservation}
  \rho T \Dt{s} = q^+ + q^- = f q^+,
\end{equation}
where 
\begin{equation}
  \label{eq:def:f}
  f = 1 - \frac{q^-}{q^+}.
\end{equation}

To solve these equations, one typically adds a relationship between
density and pressure, called an equation of state. The speed of sound
is usually defined as the adiabatic derivative
\begin{equation}
  \label{eq:def:sound:speed}
  c_s^2 = \left(\diff{\rho}{P}\right)_s,
\end{equation}
where $s$ is constant \cite{Thompson,WikipediaSpeedOfSound}.

In the case of accretion disks, we will limit ourselves to the
following reasonable equation of state \cite{Shafee08,Frank,Melia}
\begin{equation}
  \label{eq:p:rho:relation}
  P = \rho c_s^2.
\end{equation}
Note that this is equivalent to assuming the bulk modulus is equal to
the total pressure and using the Newton-Laplace equation to define the
speed of sound \cite{WikipediaSpeedOfSound}.

\subsection{The Problem of Angular Momentum Transport}
\label{subsec:angular:momentum:transport}

Now, let's talk about our fluid orbiting a Newtonian black hole. (We
will substantially weaken this assumption later.) In the following
discussion, we follow mostly \cite{Frank} and \cite{Melia}. However,
the original derivation is by Shakura and Sunyaev
\cite{ShakuraSunyaev73}.

Our first task is to describe how our black hole eats. In-falling
matter was most likely captured traveling tangential to the black
hole, not right towards it. This means that, in the center of mass
frame (essentially the frame of the black hole), the in-falling matter
carries angular momentum. Over time, may collisions with other
in-falling matter will cause the an in-falling particle to lose
vertical velocity, but the angular momentum to remains. Figure
\ref{fig:capture} shows the path of one such in-falling particle. The
particle passes the black hole with a velocity vector tangent to it
(red), so that in the frame of the black hole, the particle carries
angular momentum (blue). However, the black hole's gravitational pull
(green) captures the particle, which enters an off-axis, elliptical
orbit. Over time, collisions with other in-falling matter will remove
the vertical component of the momentum of the particle, but the
angular momentum is conserved.
\begin{figure}[htb]
  \begin{center}
    \leavevmode
    \includegraphics[scale=0.5]{../figures/capture.png}
    \caption[Capture of an in-falling particle]{Capture of an
      in-falling particle by a black hole. The particle passes the
      black hole with initial velocity tangent to the black hole
      (red). In the rest frame of the black hole, the particle
      carries angular momentum (blue). However, the acceleration due
      to the black hole's gravitational field (green) pulls the
      particle into an off-axis elliptical orbit. Over time,
      collisions with other in-falling particles circularize the orbit
      and remove the vertical component of the particle's velocity,
      leaving only the tangential in-plane velocity (purple). But the
      angular momentum remains.}
      \label{fig:capture}
  \end{center}
\end{figure}

However, if every particle in the disk possesses sufficient angular
momentum, then the black hole cannot feed. To see this, we write the
energy of a particle as a function of radius, angular momentum, and
radial velocity of the particle,
\begin{eqnarray}
  \label{eq:energy:particle}
  E &=& \frac{1}{2}m\myvec{v}^2 - \frac{G M m}{R}\\
  &=& \frac{1}{2}m v_R^2 - \frac{G M m}{R} + \frac{L^2}{2I}\\
  &\sim& \frac{1}{2}m v_R^2 - \frac{G M m}{R} + \frac{L^2}{R^2}.
\end{eqnarray}
where $M$ is the mass of the black hole, $m$ is the mass of the
particle, $\myvec{v}$ is the velocity of the particle, $v_R$ is the
radial velocity of the particle, $I$ is the moment of inertia of the
particle, $L$ is the angular momentum of the particle, and $R$ is the
radius \cite{ThortonMarion}.\footnote{We assume a Newtonian black hole
  here, but the same intuition holds in the case of a pseudo-Newtonian
  black hole too.}

Since the latter two terms are functions only of $R$, not its
conjugate variables, we can write them together as an ``effective
potential.'' Then the energy is the sum of the radial kinetic part and
the effective potential parts. If we plot the effective potential, as
in figure \ref{fig:effective:potential}, we find that it diverges as
$R\to 0$. This means that a particle needs an infinite radial kinetic
energy to reach the origin. For the black hole to eat, it must find
some mechanism to deplete the angular momentum of particles nearest
the event horizon.

\begin{figure}[htb]
  \begin{center}
    \leavevmode
    \includegraphics[scale=0.4]{../figures/effective_potential.png}
    \caption[The effective potential of a particle in a rotating
    reference frame]{The effective potential of a particle in a
      rotating reference frame. The potential diverges as the radius
      approaches zero.}
      \label{fig:effective:potential}
  \end{center}
\end{figure}

So what is the mechanism behind the angular momentum transfer? If
slower moving particles exerted a ``drag'' force on faster-moving
particles, we could get the effect we need. The Keplerian angular
velocity of a particle is inversely proportional to the radius, so
particles closer in are moving faster, and they would feel a drag
torque due to the particles further out. This torque would allow
angular momentum to be transferred radially outwards as mass and
energy were transferred radially inward
\cite{Frank,Melia,ShakuraSunyaev73}.

Viscosity exerts a force that suits our needs. However, astrophysical
fluids do not have strong enough inter-particle interactions to
justify viscosity
\cite{PadmanabhanTA,Melia,ShakuraSunyaev73}. Therefore, we need an
effective viscosity that fills the same role in the stress tensor in
equation \eqref{eq:momentum:conservation:1}.

Shakura and Sunyaev proposed that the a shear stress---and thus
angular momentum transport---could be caused by turbulent motion in
the particles in the disk \cite{ShakuraSunyaev73}, although didn't
propose a source for this turbulence. Balbus and Hawley proposed the
magneto-rotational instability (MRI) as a source for this turbulence
\cite{VanBommel,BalbusHawley1,BalbusHawley2,BalbusHawley3,BalbusHawley4},
and numerical simulations seem to demonstrate that an effective
viscosity is induced by turbulence due to magnetic fields
\cite{Melia,LehnerReview01,Font08,BucciantiniZanna13,HawleyGammieBalbus95,HawleyGammieBalbus96}.

It is worth noting that this is not the only possible source of
angular momentum transport in a disk. It is possible, for instance,
that radially-dependent magnetic fields could create a ``magnetic
pressure'' which non-locally transports angular momentum in the
appropriate way \cite{Frank,PadmanabhanTA,ShakuraSunyaev73}. Shakura
and Sunyaev originally included magnetic pressure in their
calculation, and we can likely safely treat it as included in the
$\alpha$ prescription described below \cite{ShakuraSunyaev73}.

\subsection{The Alpha Prescription}
\label{subsec:the:alpha:prescription}

We now describe how Shakura and Sunyaev use turbulent motion to attain
a shear viscosity \cite{Frank,Melia,ShakuraSunyaev73}. As we discussed
in section \ref{subsec:angular:momentum:transport}, orbits in the
accretion disk tend to circularize. If they are around a Newtonian
black hole with the standard $1/R^2$ potential, the orbits will be
Keplerian \cite{Frank,Melia,ShakuraSunyaev73}. Let's consider two
neighboring rings of fluid in the accretion disk which follow such
orbits. Call them ring $A$ and ring $B$. Let ring $A$ be at radius $R$
with angular velocity $\Omega(R)$ and let ring $B$ be at radius
$R+\lambda$ with angular velocity $\Omega(R+\lambda)$, as shown in
figure \ref{fig:disk:neighboring:rings}.

\begin{figure}[htb]
  \begin{center}
    \leavevmode
    \begin{picture}(11,4)
      \put(0,0){\includegraphics[scale=0.6]{../figures/accretion_disk_alpha_approximation.pdf}}
      \put(3,0.1){\huge $A$}
      \put(10.6,0.1){\huge $B$}
      \put(0.9,1.1){\LARGE $ba$}
      \put(7.95,0.35){\LARGE $ab$}
      \put(4.5,1.4){\large $v_\phi^{A} = R\Omega(R)$}
      \put(3.5,2.9){\large $v_\phi^{B} = (R+\lambda)\Omega(R+\lambda)$}
    \end{picture}
  \end{center}
  \caption[Turbulence as shear viscosity]{Turbulence as shear
    viscosity. Two rings in the accretion disk near radius $R$. A
    clump of matter from ring $A$ moves to $B$. Call it $ab$. Because
    the fluid is in equilibrium, at the same time, a clump of matter
    from ring $B$ moves to ring $A$. Call it $ba$. Ring $A$ moves with
    azimuthal velocity $v_\phi^A = R\Omega(R)$. Ring $B$ moves with
    azimuthal velocity $v_\phi^B = (R+\lambda
    \Omega(R+\lambda)$. Image inspired by those in
    \cite{Frank,Melia}.}
  \label{fig:disk:neighboring:rings}
\end{figure}

Because of turbulence, clumps of fluid travel between the rings at
random, each one carrying angular momentum. Since the fluid
is in local hydrodynamic equilibrium, the net flow of clumps is
zero---there are as many clumps moving radially outward as there are
moving radially inward \cite{Frank,Melia}. Each clump travels a
distance of approximately $\lambda$ at a radial speed of
$\tilde{v}$. Each clump has a tangential velocity equal to its radius
times its angular velocity. Let's consider two such clumps. One going
from ring $A$ to ring $B$, call it $ab$ and one going from ring $B$ to
ring $A$, call it $ba$. $ab$ has tangential velocity $v_\phi^{A} =
R\Omega(R)$ and $ba$ has tangential velocity
$v_\phi^{B}=(R+\lambda)\Omega(R+\lambda)$ \cite{Frank,Melia}.

As a simplifying assumption, we let the tangential linear angular
momentum be conserved.\footnote{If we had instead allowed the angular
  momentum to be conserved, we still would get angular momentum
  transport radially outward, which is what we want \cite{Frank}.}
Then the tangential linear momentum flux per unit arc length at a
constant radius $R+\lambda/2$ in the outward direction is \cite{Frank}
\begin{eqnarray}
  \Phi_\phi^{AB} &=& 2\rho \tilde{v} H (R+\lambda) v_\phi^A\nonumber\\
  \label{eq:tangential:linear:momentum:flux:outward}
  &=& 2\rho \tilde{V} H (R+\lambda)R\Omega(R),
\end{eqnarray}
where $H$ is the height of the disk, as in equation
\eqref{eq:variable:definitions:9}. Similarly, the inward flux is
\cite{Frank}
\begin{equation}
  \label{eq:tangential:linear:momentum:flux:inward}
  \Phi_\phi^{BA} = 2\rho\tilde{v} H R (R+\lambda) \Omega(R+\lambda).
\end{equation}
Thus the torque on the outer ring by the inward ring is given by the
net (outward) flux, which we can approximate as the sum of equations
\eqref{eq:tangential:linear:momentum:flux:outward} and
\eqref{eq:tangential:linear:momentum:flux:inward}. In the small
$\lambda$ limit, the difference in angular momenta becomes a
derivative and we get \cite{Frank,Melia}
\begin{equation}
  \label{eq:torque:AB}
  \tau_{AB} = -\rho\tilde{v}H\lambda R^2 \dR{\Omega(R)}.
\end{equation}
Notice that the torque goes to zero in the case of rigid motion where
$\dR{\Omega}$ is zero. Furthermore, if the angular velocity decreases
radially, then the torque outward is positive, transporting angular
momentum radially outward \cite{Melia}. These are the limits we want.

Then the azimuthal shear stress, or the force in the azimuthal
direction per unit area:
\begin{equation}
  \label{eq:azimuthal:shear:stress}
  T_{\phi} = \rho\tilde{v}\lambda R\dR{\Omega(R)}.
\end{equation}
Or, the only non-zero component of the stress tensor is the
radial-azimuthal component.
\begin{equation}
  \label{eq:stress:tensor:nonzero:1}
  \sigma_{R\phi} = \rho\tilde{v}\lambda R\dR{\Omega(R)}.
\end{equation}

We don't know the details of $\lambda$ or $\tilde{v}$, so we must make
some guesses. The motion of the clumps comes from turbulence, and the
turbulent eddies are unlikely to be larger than the disk height. So we
know \cite{Frank,Melia}
\begin{equation}
  \label{eq:lambda:bound}
  \lambda \leq H.
\end{equation}
Similarly, it is unlikely that the clumps are moving greater than the
speed of sound. So we know that \cite{Frank,Melia}
\begin{equation}
  \label{eq:v:tilde:bound}
  \tilde{v} \leq c_s.
\end{equation}
We then parametrize our lack of knowledge about $\lambda$ and
$\tilde{v}$ by the parameter $\alpha \in [0,1]$ \cite{Frank,Melia}
\begin{equation}
  \label{eq:stress:tensor:nonzero:alpha}
  \sigma_{R\phi} = \alpha \rho c_s H R\dR{\Omega(R)}.
\end{equation}
This is the famous $\alpha$ prescription of Shakura and Sunyaev
\cite{ShakuraSunyaev73}. It's important to note, however, that we've
simply hidden what we don't know inside the $\alpha$ parameter, which
we determine numerically. We haven't explained what $\alpha$ should be
\cite{Frank,Melia}. In the language of mechanical viscosity, this
description becomes
\begin{equation}
  \label{eq:alpha:prescription:viscosity}
  v = \tilde{v}\lambda = \alpha c_s H.
\end{equation}

Shakura and Sunyaev actually took this a little further. They argued
that the azimuthal velocity is approximately constant and thus
\begin{equation}
  \label{eq:omega:dependence:constant:vphi}
  \dR{\Omega} \approx - \frac{1}{R} \Omega.
\end{equation}
This isn't as crazy as it sounds, since it is approximately true in the
Keplerian case \cite{ShakuraSunyaev73,ThortonMarion}. Now, in the
case of a Newtonian potential, it is easy to show that, up to factors
of order 1,
\begin{equation}
  \label{eq:H:def}
  H = \frac{c_s}{\Omega_K}
\end{equation}
(see appendix \ref{app:H:relation}). If the orbits are all Keplerian
anyway, then 
$$H = \frac{c_s}{\Omega}$$
and we can use equations \eqref{eq:omega:dependence:constant:vphi} and
\eqref{eq:stress:tensor:nonzero:alpha} to find
\begin{equation}
  \label{eq:alpha:prescription:rho}
  \sigma_{R\phi} = -\alpha \rho c_s^2.
\end{equation}
Or, if we use equation \eqref{eq:p:rho:relation} \cite{Shafee08},
\begin{equation}
  \label{eq:alpha:prescription:P}
  \sigma_{R\phi} = -\alpha P.
\end{equation}

\subsection{Radiation}
\label{subsec:radiation}

Let's return to the outward torque we calculated in section
\ref{subsec:the:alpha:prescription}. The torque is radius dependent,
so we can write \cite{Frank,Melia}
\begin{equation}
  \label{eq:d:tau}
  \tau_{out}(R) - \tau_{out}(R+dR) = - \dR{\tau_{out}}dR,
\end{equation}
where $\tau_{out}$ is just $\tau_{AB}$ from equation
\eqref{eq:torque:AB} rewritten with $\alpha$,
\begin{equation}
  \label{eq:tau:out}
  \tau_{out}(R) = -\alpha\rho c_s H^2 R^2 \dR{\Omega(R)}.
\end{equation}
Then the power dissipated by this infinitesimal torque is
\cite{Frank,Melia}
\begin{equation}
  \label{eq:power:dissipated}
  P = -\Omega \dR{\tau_{out}} dR.
\end{equation}
But by the product rule,
\begin{equation}
  \label{eq:product:rule:tau}
  \ddR(\tau_{out}\Omega) = \Omega \dR{\tau_{out}} + \tau \dR{\Omega}.
\end{equation}
So \cite{Melia}, 
\begin{equation}
  \label{eq:P:pieces}
  P = - \left[ \ddR (\tau_{out}\Omega) - \tau_{out}\dR{\Omega}\right]dR.
\end{equation}
The first term gives the transfer of rotational energy from the inner
radius of the disk to the outer radius of the disk \cite{Melia}. If we
integrate this term, we simply get the difference in rotational energy
between the inner-most ring at $R_{inner}$ and the outer ring at
$R_{outer}$ \cite{Melia}:
\begin{equation}
  \label{eq:difference:rotational:energy}
  \int_{R_{inner}}^{R_{outer}} \ddR(\tau_{out}\Omega)dR = \tau_{out}\Omega\eval_{R_{outer}} - \tau_{out}\Omega\eval_{R_{inner}}.
\end{equation}
The other term, however, represents the local dissipation of energy
into heat \cite{Melia}. Ultimately, this energy will be radiated by
the upper and lower faces of the disk, so in anticipation of a
luminosity calculation, we define the energy dissipated per time per
unit plane area. Each ring of the disk is at radius $R$ with
circumference $2\pi R$ and width $d$. And it has two surfaces. So we
have a unit plane area of
\begin{equation}
  \label{eq:unit:plane:area}
  4\pi R dR
\end{equation}
and dissipated energy per unit area per time of \cite{Frank,Melia}
\begin{equation}
  \label{eq:def:D(R)}
  D(R) = \frac{\tau_{out}}{4\pi R}\dR{\Omega}.
\end{equation}

The luminosity per unit area at a given radius will be defined by the
cooling rate of the fluid (see equation \eqref{eq:def:f}) times
$D(R)$ \cite{Shafee08}:
\begin{equation}
  \label{eq:radiant:power:disk}
  j^*(R) = (1-f) D(R).
\end{equation}
We can use the Stefan-Boltzmann law to solve for the ``effective
temperature'' of the disk at a given radius:
\begin{equation}
  \label{eq:stefan:boltzmann:law}
  j^*(R) = \sigma_B T^4(R) \text{ with } \sigma_B = \frac{2\pi^5 k_B}{15c^2 h^3},
\end{equation}
where $k_B$ is Boltzmann's constant, $c$ is the speed of light in
vacuum, and $h$ is Planck's constant \cite{Melia}. If we assume
blackbody radiation, this gives us the radiated spectrum at a given
temperature. The intensity of a given frequency $\nu$ is
\cite{TaylorZafiratosDubson}
\begin{equation}
  \label{eq:planck:spectrum}
  I(\nu,R) = \frac{2 h\nu^3}{c^2} \frac{1}{e^{h\nu}{k_BT(R)}-1}.
\end{equation}
And, if we want the observed intensity, we average over every radius
and adjust the normalization for distance $D_{EBH}$ and viewing angle
$i$ \cite{Shafee08}:
\begin{equation}
  \label{eq:observed:spectrum}
  F(\nu) = \frac{2\pi \cos(i)}{D_{EBH}^2} \int_{R_{inner}}^{R_{outer}} \frac{2 h \nu^3 R}{c^2 e^{h\nu/k_B T(R)} -1}dR.
\end{equation}

\section{The Shafee Model}
\label{sec:the:model}

Now we want to use the tools of section \ref{sec:background} to model a
quasi-realistic accretion disk, where the fluid flow can go from
subsonic velocities to supersonic velocities.

In the standard approach, one can find the inner-most stable circular
orbit (ISCO) around a fully-relativistic fixed Schwarzschild or Kerr
black hole\footnote{It is perhaps a surprising feature of black holes
  that they can have an inner-most stable circular orbit outside of
  the event horizon \cite{Melia,Wald}.} and assume that the viscous
torque, angular momentum loss, and energy dissipation are all zero
within the ISCO \cite{Zhang97,Shafee06,McClintock06,Davis05}. From
there one can use the hydrodynamics equations and a wide variety of
boundary conditions to solve for the properties of an axisymmetric
thin disk.

To test whether or not it is valid to assume these boundary conditions
at the ISCO, Shafee et al. developed a pseudo-Newtonian model that
still captures some of the relativistic flavor and all of the
trans-sonic flavor. The following analytic model of thin accretion
disks was pioneered by Paczynski and Bisnovatyi-Kogan \cite{PBK81} and
developed by many others
\cite{Shafee08,MuchotrzebPaczynski82,Kato88a,Abramowicz88,PophamNarayan91,NarayanPopham93,ChenTaam93,Narayan97,Chen97,AfshordiPaczynski03}. It
is less accurate than a fully relativistic simulation, but it is more
accurate than the standard thin disk model \cite{Shafee08}. Since we
are reproducing their results, we primary follow the treatment of
Shafee et al. \cite{Shafee08} who were in turn guided by Narayan et
al. \cite{Narayan97}.

\subsection{The Pseudo-Newtonian Black Hole}
\label{subsec:gravity}

Because we want to avoid the extreme computational cost of a fully
relativistic magnetohydrodynamics simulation, we will study a simple
viscous fluid in a Newtonian gravitational potential. The standard
disk model assumes the existence of an ISCO, and we would like to make
contact with this notion. For that reason, we use a pseudo-Newtonian
gravitational potential, which posses such a radius.

Before we proceed, it is convenient to define the following quantities. Let
\begin{eqnarray}
  \label{eq:convenient:definitions}
  R_g &=& \frac{GM}{c^2}\text{ is twice the Schwarzschild radius of the black hole}\\
  \text{and }a_* &=& \frac{J c}{G M^2}\text{ s.t. }-1<a_*<1\text{ is the dimensional spin of the black hole},
\end{eqnarray}
where $M$ is the mass of the black hole, $J$ is the spin of the black
hole, $c$ is the speed of light, and $G$ is Newton's
constant.\footnote{We can write the spin of the black hole in
  mass-energy terms as $a=a_* M$ \cite{Shafee08}.}

So that we may include spin, we use Mukhopadhyay's pseduo-Kerr
proposal \cite{Mukhopadhyay02}, where the acceleration due to gravity
of a test particle in a Keplerian orbit at a distance $R$ from the
black hole singularity is\footnote{We are writing all of the following
  formulas in the notation of \cite{Shafee08}.}
\begin{equation}
  \label{eq:mukhopadhyay:g}
  g = -\myvec{\nabla}\Phi = \frac{c^4}{GM} \frac{\left(r^2 - 2 a_* \sqrt{r} + a_*^2\right)^2}{r^2\left(\sqrt{r}(r-2)+a_*\right)^2},
\end{equation}
where $\Phi$ is the gravitational potential and $r=R/R_g$.

When we work with the fluid dynamics, we will hide the effects of
gravity in the Keplerian angular velocity at a given radius. (See
section \ref{subsec:keplerian:motion}.) In Mukhopadhyay's model, this
is \cite{Shafee08}:
\begin{equation}
  \label{eq:mukhopadhyay:omega:k}
  \Omega_K = \frac{c^3}{GM}\frac{(r^2 - 2a_* \sqrt{r} + a_*^2)}{r^2\left(\sqrt{r}(r-2)+a_*\right)}.
\end{equation}
In the case that  $a_*=0$, the Keplerian angular velocity reduces to 
\begin{equation}
  \label{eq:PW80:omega:k}
  \Omega_k = \frac{1}{R-2R_g}\sqrt{\frac{GM}{R}}.
\end{equation}
This is the Keplerian angular velocity associated with the
gravitational potential proposed by Paczynski and Wiita, also known as
the PW80 potential \cite{PW80}:
\begin{equation}
  \label{eq:PW80:phi}
  \Phi = - \frac{GM}{R-2R_g}.
\end{equation}
We won't need equations \eqref{eq:mukhopadhyay:g} or
\eqref{eq:PW80:phi}. All the information we need will be contained in
equations \eqref{eq:mukhopadhyay:omega:k} and \eqref{eq:PW80:omega:k}.

\subsection{The Accreting  Fluid}
\label{subsec:hydro}

To avoid the complexities of (3+1)-dimensional fluid dynamics, we
assume axisymmetry and hydrostatic equilibrium for the fluid in the
accretion disk. These assumptions are obviously not accurate, however
more general simulations and some observational evidence suggest that
in the case of a very thin disk, they are often good enough
\cite{Shafee06,McClintock06,Liu08,Gou09,Penna10,PennaThesis}.

Since we are assuming hydrostatic equilibrium, the co-moving
derivative operator \eqref{eq:comoving:derivative} takes an especially
simple form:
\begin{equation}
  \label{eq:comoving:derivative:2}
  \DDt = \pdd{t} + \myvec{v}\cdot\myvec{\nabla} = v_R \dd{R},
\end{equation}
where $t$ is time \cite{Shafee08}. We also assume the $\alpha$
prescription of of Shakura and Sunyaev \cite{ShakuraSunyaev73},
specifically in the form of equation \eqref{eq:alpha:prescription:P}
so that the only component of the viscous stress tensor is the
radial-azimuthal shear stress, which is proportional to the
pressure. Similarly, as a simplifying assumption we assume that the
height is approximately the speed of sound divided by the Keplerian
angular velocity as in equation \eqref{eq:H:def}.

Now we will endeavor to solve the equations of hydrodynamics, which
consist of a mass conservation equation \eqref{eq:continuity:1}, a
momentum conservation equation \eqref{eq:momentum:conservation:1}, and
an energy conservation equation, which we will write as an entropy
conservation equation \eqref{eq:entropy:conservation}.

Between radii $R$ and $R+\Delta R$, the mass is \cite{Melia}
\begin{equation}
  \label{eq:mass}
  \text{mass} = 4 \pi R \rho H dR.
\end{equation}
If we differentiate this equation with respect to time, we find that
\cite{Melia}
\begin{equation}
  \label{eq:continuity:2}
  R \diff{4\pi\rho H}{t} + \ddR\left(4\pi R\rho H v_R\right) = 0.
\end{equation}
And, since we are in a steady state, we find that
\begin{equation}
  \label{eq:continuity:3}
  \ddR\left(4\pi R\rho H v_R\right) = 0.
\end{equation}
If we integrate, we find that $-\pi R\rho H v_R$ is conserved with
respect to the radius. We call this quantity our accretion rate,
$\dot{M}$ such that
\begin{equation}
  \label{eq:continuity:final}
  \dot{M} = -4\pi \rho v_R R H = \text{ constant.}
\end{equation}
This is the integrated form of the continuity equation defined in
equation \eqref{eq:continuity:1} \cite{Shafee08,Melia}.

% If we feed equation \eqref{eq:comoving:derivative:2} and axisymmetry
% into the continuity equation (equation \eqref{eq:continuity:1}), we
% have
% \begin{equation}
%   \label{eq:continuity:2}
%   v_R \dR{\rho} +  \rho \dR{v_R} = 0.
% \end{equation}
% We can recognize this as a product rule and write
% \begin{equation}
%   \label{eq:continuity:3}
%   \ddR (v_R \rho) = 0.
% \end{equation}
% If we integrate, we have
% \begin{equation}
%   \label{eq:continuity:4}
%   v_R \rho = \text{constant}.
% \end{equation}
% We can also integrate over the thickness of the disk to find that
% \begin{equation}
%   \label{eq:continuity:5}
%   2Hv_R \rho = \text{constant}.
% \end{equation}
% Finally, because the mass between two

With the assumptions of axisymmetry, stead-state, and equations With
the addition of equations \eqref{eq:keplerian:angular:velocity},
\eqref{eq:p:rho:relation}, \eqref{eq:H:def} and
\eqref{eq:alpha:prescription:P}, we can break the momentum
conservation equation (equation \eqref{eq:momentum:conservation:1})
into a radial part and an azimuthal part. The radial part becomes
\cite{Shafee08}
\begin{equation}
  \label{eq:radial:momentum}
  v_R \dR{v_R} = - (\Omega_K^2 - \Omega^2)R - \frac{1}{\rho}\ddR(\rho c_s^2).
\end{equation}
Rather than look at the azimuthal part, we impose conservation of
angular momentum directly to find that \cite{Shafee08}
\begin{equation}
  \label{eq:angular:momentum:conservation}
  \frac{\rho v_R}{R} \ddR (\Omega R^2) =  \frac{1}{R^2 H} \diff{d(R^2 H \sigma_{R\phi})}{dR},
\end{equation}
where $\Omega R^2$ is the angular momentum per unit mass of the disk at
a given radius. This equation can be integrated to find that
\begin{equation}
  \label{eq:integrated:angular:momentum}
  j = \Omega R^2 + \frac{\alpha c_s^2 R}{v_R},
\end{equation}
where $j$ is an integration constant \cite{Shafee08}. 

To adapt the entropy equation (equation
\eqref{eq:entropy:conservation}) to our needs, we recall that the
heating of the gas is due to viscous dissipation (see equation
\eqref{eq:def:D(R)}). This gives us that
\begin{equation}
  \label{eq:entropy:change:2}
  \rho T \Dt{s} = - f \alpha \rho c_s^2 R \dR{\Omega},
\end{equation}
where here we used equations \eqref{eq:def:sound:speed} and
\eqref{eq:tau:out} \cite{Shafee08}.

Now we want to set the right-hand side of equation
\eqref{eq:entropy:change:2} equal to the change in entropy of a
comoving clump of gas in the disk.  To study our comoving clump of
gas, we keep the mass of the gas fixed but allow the other
thermodynamic quantities to vary. We will use the fundamental
thermodynamic relation
\begin{equation}
  \label{eq:fundamental:thermodynamic:relation}
  dU = T dS - P dV,
\end{equation}
where $U$ is the internal energy of the system, $S$ is the entropy of
the system, and $V$ is the volume \cite{Schroeder}. If we reformulate
this in the context of fluid dynamics, we have
\begin{equation}
  \label{eq:fundamental:thermodynamic:relation:fluids}
  \rho T \Dt{s} = \Dt{\epsilon} + \frac{P}{V} \Dt{V},
\end{equation}
where $\epsilon$ is the energy per unit volume and $s$ is the entropy
per unit mass (also called specific entropy).

Let's look at the right-most term first. Because of equation
\eqref{eq:comoving:derivative:2}, we have that
\begin{equation}
  \label{eq:Dt:V}
  \frac{P}{V}\Dt{V} = v_R\frac{P}{V}\dR{V}.
\end{equation}
Now, because mass is held fixed, we can write a relationship between
$\dR{\rho}$ and $\dR{V}$:
\begin{eqnarray}
  \dR{\rho} &=& \mu\ddR{V^{-1}}\nonumber\\
  &=& -\mu V^{-2} \dR{V}\nonumber\\
  \label{eq:d:rho:d:R}
  \Rightarrow \frac{1}{V}\dR{V} &=& - \frac{1}{\rho}\dR{\rho},
\end{eqnarray}
where $\mu$ is the mass of the clump of gas. So, if we also use
equation \eqref{eq:p:rho:relation}, equation \eqref{eq:Dt:V} becomes
\begin{equation}
  \label{eq:eq:Dt:V:final}
  \frac{P}{V} \Dt{V} = - c_s^2 v_R \dR{\rho}.
\end{equation}

Now let's look at the $\Dt{\epsilon}$ term in equation
\eqref{eq:fundamental:thermodynamic:relation:fluids}. This term should
be independent of the volume, so we hold both mass and volume
fixed. If we assume the ideal gas law for our fluid,
\begin{equation}
  \label{eq:idea:gas:law}
  PV = N k_B T,
\end{equation}
where $V$ is volume and $N$ is the number of molecules
\cite{Schroeder}, and we assume our gas has an adiabatic index
$\gamma$,\footnote{We follow Shafee et al. and assume $\gamma=3/2$
  \cite{Shafee08}.} then at a constant volume $V$, the heat capacity
is
\begin{equation}
  \label{eq:heat:capacity}
  C_V = \frac{N K_B}{\gamma - 1},
\end{equation}
and the thermal energy per unit volume is
\begin{equation}
  \label{eq:thermal:energy:per:unit:volume}
  \epsilon = \frac{C_V T}{V} = \frac{P}{\gamma - 1}.
\end{equation}
So,
\begin{equation}
  \label{eq:derivative:energy:per:unit:volume}
  \Dt{\epsilon} = \frac{v_R}{\gamma - 1} \left(\rho\dR{c_s^2} + c_s^2 \dR{\rho}\right).
\end{equation}
However, with mass and volume fixed, $\dR{\rho} = 0$ and we have that
\begin{equation}
  \label{eq:derivative:energy:per:unit:volume:final}
  \Dt{\epsilon} = \frac{v_R\rho}{\gamma - 1}\dR{c_s^2}.
\end{equation}

Now we can combine equations
\eqref{eq:fundamental:thermodynamic:relation:fluids},
\eqref{eq:eq:Dt:V:final} and
\eqref{eq:derivative:energy:per:unit:volume:final} to get the change
in entropy for a clump of gas in the disk of fixed mass
\cite{Shafee08,Narayan97}:
\begin{equation}
  \label{eq:entropy:conservation:heat:v:cool}
  \rho T \Dt{s} = \frac{\rho v_R}{\gamma - 1} \dR{c_s^2} - c_s^2 v_R \dR{\rho}.
\end{equation}
And, finally, we set equations \eqref{eq:entropy:change:2} and
\eqref{eq:entropy:conservation:heat:v:cool} equal to each other and we
find the energy conservation equation for Shafee et al.'s model \cite{Shafee08}:
\begin{equation}
  \label{eq:energy:conservation:final}
  \frac{\rho v_R}{\gamma - 1}\dR{c_s^2} - c_s^2 v_R \dR{\rho} = - f \alpha \rho c_s^2 R \dR{\Omega}.
\end{equation}

\subsection{The First-Order ODE System}
\label{subsec:first:order:ODE:system}

Other than boundary conditions, we've now written down all the
ingredients of Shafee et al.'s model. It would be convenient to reduce
the numerous equations we have to a system of ODES to solve for just a
few functions of $R$. We use the definitions of the speed of sound \eqref{eq:p:rho:relation} and H \eqref{eq:H:def} and the continuity
equation \eqref{eq:continuity:final} to reduce the variables in the
radial \eqref{eq:radial:momentum} and angular
\eqref{eq:angular:momentum:conservation} momentum and energy equations
\eqref{eq:energy:conservation:final} to just $R$ and three functions
of $R$: $v_R$, $c_s^2$, and $\Omega$. We find that
\begin{eqnarray}
  \label{eq:ODE:system:dv}
  \ddR v_R &=& \frac{v_R}{2\Gamma} \left[\Omega_K \left(3 (\gamma-1) \chi + 2 c_s^2\Sigma - v_R^2 L (\gamma+1)\right)
    -2 R c_s^2 \Omega_k' \left(f \alpha^2 c_s^2 (\gamma-1) - v_R^2\gamma\right)\right]\\
  &&\nonumber\\
  \label{eq:ODE:system:dc2s}
  \ddR c_s^2 &=& \frac{(\gamma-1)c_s^2}{\Gamma}\left[\Omega_K\left(2\chi + f\psi\left(\alpha L 
        + 2 R v_R\Omega -\alpha v_R^2\right) + v_R^2 \Xi\right)
    - R \Omega_K'\left(\chi+v_R^4\right)\right]\\
  &&\nonumber\\
  \label{eq:ODE:system:domega}
  \ddR \Omega &=& -\frac{v_R}{2\Gamma R}\left[\Omega_K \left(4\alpha (c_s^2)^2 \gamma +c_s^2\Lambda - 2 R v_R^3 \Omega(\gamma+1)\right) - 2\alpha R c_s^2 \Omega_K'\left(\gamma c_s^2 + v_R^2(\gamma-1)\right)\right]
\end{eqnarray}
with 
\begin{eqnarray}
  \label{eq:def:psi}
  \psi &=& \alpha c_s^2\\
  \label{eq:def:chi}
  \chi &=& f\psi^2\\
  \label{eq:def:L}
  L &=& R^2\left(\Omega^2 - \Omega_K^2\right)\\
  \label{eq:def:Xi}
  \Xi &=& v_R^2 - 2 f \alpha R v_R \Omega + L\\ %v_R^2 + 2 f \alpha R v_R \Omega + L\\
  \label{eq:def:sigma}
  \Sigma &=& f\alpha R\Omega v_R(\gamma -1) -v_R^2 \gamma - f\alpha^2 (\gamma-1)L\\
  \label{eq:def:lambda}
  \Lambda &=& \alpha v_R^2(\gamma-3) + 4 R\Omega v_R \gamma + \alpha L \left(3\gamma - 1\right)\\
  \label{eq:def:Gamma}
  \Gamma &=&\frac{1}{2} R \Omega_K \left[2v_R^2\left(f\alpha^2 (\gamma -1)+\gamma\right) - \chi(\gamma-1) - v_R^4(\gamma+1)\right]
\end{eqnarray}
% \begin{eqnarray}
%   \label{eq:ODE:system:dv}
%   \ddR v_R &=& \frac{v_R\left[\Omega_K\left(v_R^2 R^2 \Omega^2 + c_s^2 (2 f \alpha v_R R \Omega + 2 v_R^2 + c_s^2 f \alpha^2) - v_R^2 R^2 \Omega_K^2\right) - 2 v_R^2 \Omega_k' R c_s^2\right]}{R\Omega_K\left[v_R^4 + c_s^2\left(f\alpha^2 c_s^2 - 2 v_R^2\right)\right]}\\
%   &&\nonumber\\
%   \label{eq:ODE:system:dc2s}
%   \ddR c_s^2 &=& \frac{2 c_s^2}{R\Omega_K\left[v_R^4 + c_s^2\left(f\alpha^2 c_s^2 - 2 v_R^2\right)\right]}\left[R\Omega_K'(v_R^4 - c_s^4 f \alpha^2) + \Omega_K\left(R^2\Omega^2(c_s^2 f \alpha^2 -v_R^2)\right. \right.\\
%   &&\qquad  \left. \left.- v_R^2 (v_R^2 + 2 f \alpha v_R R \Omega + c_s^2f\alpha^2) + 2 f \alpha c_s^2 (\alpha c_s^2 + v_R R \Omega) - R^2 \Omega_K^2 (c_s^2 f \alpha^2 - v_R^2)\right)\right]\nonumber\\
%   &&\nonumber\\
%   \label{eq:ODE:system:domega}
%   \ddR \Omega &=& \frac{v_R\left[\Omega_K\left(c_s^2(4\alpha c_s^2 + 4 v_R R \Omega) - v_R^2 \alpha c_s^2 - 2 v_R^3 R \Omega + \alpha c_s^2 R^2 (\Omega^2 - \Omega_K^2)\right) - 2 \alpha c_s^2 R \Omega_K'\right]}{R^2\Omega_K\left[v_R^4 + c_s^2\left(f\alpha^2 c_s^2 - 2 v_R^2\right)\right]}
% \end{eqnarray}
where $\Omega_K$ as a function of $R$ is given by equation
\eqref{eq:mukhopadhyay:omega:k} and we can easily find
$\Omega_K'=\ddR\Omega_K$ by differentiating $\Omega_K$.

Note that if we write
\begin{equation}
  \label{eq:y:definition}
  \myvec{y}(R) = \threevector{v_R(R)}{c_s^2(R)}{\Omega(R)},
\end{equation}
then we can rewrite our ODE system as a single vector-valued ODE
\begin{equation}
  \label{eq:y:prime:equation}
  \ddR\myvec{y} = \myvec{O}(y,R),
\end{equation}
for the appropriate vector-valued function $\myvec{O}$.

\subsection{Boundary Conditions and the Shooting Method}
\label{subsec:boundary:conditions}

We now have a system of three coupled first-order ODEs and three
unknown functions of one variable. To solve the system we need the
following:
\begin{itemize}
\item A length scale. Over what range of $R$ do we wish to solve the
  system?
\item Boundary data. We need three boundary conditions to completely
  specify the system.
\end{itemize}

It turns that these two issues are intimately related. Ultimately, we
want to study the disk from the Schwarzschild radius to the edge of
the disk. However, if we allow spherically-symmetric (Bondi) accretion
to inform our intuition, we should expect fluids far from the black
hole to have speeds less than the speed of sound and fluids near the
black hole to have speeds exceeding the speed of sound
\cite{Frank,PadmanabhanTA,Melia}. 

The point where the fluid speed exceeds the sound speed is called the
sonic point, $R_s$, and we should expect our ODE system to become
degenerate at this point
\cite{Shafee08,Frank,PadmanabhanTA,Melia,Narayan97}. To deal with this
issue, we will impose regularity conditions that force our ODE system
to be well-behaved at the sonic point and use this as boundary data. We
can then numerically integrate equations \eqref{eq:ODE:system:dv},
\eqref{eq:ODE:system:dc2s}, and \eqref{eq:ODE:system:domega} inward from the boundary of the disk to the sonic point. At the sonic point, we check to insure the regularization conditions hold. If they do, we continue to integrate to the Schwarzschild radius. \cite{Shafee08}.

There's only one hangup. We don't know where the sonic point
\textit{is}. To solve this problem, we use some clever guesswork. We
guess a value for $R_s$ and integrate inward to it. If the values of
$v_R$, $c_s^2$, and $\Omega$ at $R_s$ match the regularization
conditions, then $R_s$ is the correct sonic point. We can now
integrate inwards to the Schwarzschild radius to find the full
solution. Otherwise, we cleverly guess a new value for $R_s$ and try
again. This is called the \textit{shooting method}
\cite{NumericalRecipes,Heath}.

\subsubsection{The Self-Similar Solution}
\label{subsubsec:self-similar}
Let's talk about the boundary data at the outer boundary of the
disk. Since the gravitational potential disappears far from the black
hole, it's reasonable to guess that the accretion flow would asymptote
to some uniform behavior far from the black hole. In 1987, Spruit et
al. described a set of self-similar solutions for accretion disks
while studying shocks in the disk \cite{Spruit87}. In 1994, Narayan
and Yi rediscovered this solution family. In 1997, Narayan et
al. found numerically that, far from the disk, solutions do indeed
tend to be self-similar \cite{Narayan97}.

Assume that
\begin{equation}
  \label{eq:self:similar}
  \rho = a R^{-3/2},\ v_R = b R^{-1/2},\ \Omega = c R^{-3/2},\text{ and }c_s^2 = d R^{-1},
\end{equation}
where $a$, $b$, $c$, and $d$ are constants
\cite{NarayanYi94}. Plug this ansatz into the accretion flow
equations \eqref{eq:ODE:system:dv}, \eqref{eq:ODE:system:dc2s}, and
\eqref{eq:ODE:system:domega}, and solve for $a$, $b$, $c$, and $d$. Shafee
et al. followed this procedure and found the following self-similar
solution \cite{Shafee08}:\footnote{We follow Shafee et al.'s lead and
  use the ``ss'' subscript to mean ``self-similar'' \cite{Shafee08}.}
\begin{eqnarray}
  \label{eq:self:similar:c2s}
  c_{s,ss}^2(R) &=& c_s^2 \frac{GM}{R} \text{ where } c_0^2 = \frac{2\epsilon '}{5\epsilon ' + 2(\epsilon ')^2 + \alpha^2}\text{ and }\epsilon ' = \frac{5/3-\gamma}{f(\gamma-1)},\\
  \label{eq:self:similar:v:R}
  v_{R,ss}(R) &=& v_0\sqrt{\frac{GM}{R}},\text{ where }v_0 = -\alpha\sqrt{\frac{c_0^2}{\epsilon '}},\\
  \label{eq:self:similar:omega}
  \Omega_{ss}(R) &=& \Omega_0\Omega_K(R),\text{ where }\Omega_0=\sqrt{c_0^2\epsilon_0}.
\end{eqnarray}
It is easy to reproduce this calculation in just a few lines in the
likes of Maple or Mathematica. 

Because Narayan et al. demonstrated that solutions tend to become self
similar \cite{Narayan97}, rather than solving for the boundary
conditions at the edge of the disk, we simply force the solution to be
self-similar far away from the sonic point. We call the outer boundary
of our computational domain $R_{out}$ and set it equal to
\begin{equation}
  \label{eq:def:R:out}
  R_{out} = 10^5 R_s.
\end{equation}
Then we have the following outer boundary conditions:
\begin{eqnarray}
  \label{eq:vr:rout}
  v_R(R_{out}) &=& v_0 \sqrt{\frac{GM}{R_{out}}}\\
  c_s^2(R_{out}) &=& c_0^2 \frac{GM}{R_{out}}\\
  \Omega(R_{out}) &=& \Omega_0\Omega_K.
\end{eqnarray}

\subsubsection{Asymptotic Disk Thickness}
\label{subsubsec:asymptotic:disk:thickness}

Before we continue to the conditions at the sonic point, we take a
brief detour to discus the asymptotic behavior of the thickness of the
disk. Asymptotically,
\begin{equation}
  \label{eq:omega:k:self-similar}
  \Omega_K = \sqrt{\frac{GM}{R}}.
\end{equation}
And so \cite{Shafee08}
\begin{eqnarray}
  \label{eq:H:R}
  H &=& \frac{c_{s,ss}}{\Omega_K} = c_0 R \nonumber\\
  \Rightarrow \frac{H}{R} &=& \sqrt{2} \left( 5+2\,{\frac {5/3-\gamma}{f \left( \gamma-1
 \right) }}+{\frac {{\alpha}^{2}f \left( \gamma-1 \right) }{5/3-\gamma
}} \right) ^{-1/2}.
\end{eqnarray}
This means that we can set the asymptotic height scale of the disk by
setting $f$ \cite{Shafee08}. As we will see numerically, the height
scale does not change dramatically over most of the disk, so we can
enforce that we have a thin disk \cite{Shafee08}. For $\gamma = 1.5$,
$f = 3.5\times 10^{-3}$ gives $H/R = 0.1$ and $f=3.5\times 10^{-5}$
gives $H/R=0.01$. We choose $H/R = 0.01$, which enforces an extremely
thin dramatically radiation-cooled disk.

\subsubsection{Regularizing the Sonic Point}
\label{subsubsec:regularizing:the:sonic:point}

When $\|\myvec{v}\| = \sqrt{\left(v_R \hat{R} + R
    \Omega(R)\hat{\phi}\right)^2}$ becomes equal to $c_s^2$, our ODE
system breaks down. To force it to stay well behaved, we impose
regularity conditions as boundary conditions. It will be sufficient to
force the derivative of the radial velocity to behave correctly, we
substitute the energy equation \eqref{eq:energy:conservation:final}
into the radial momentum conservation equation
\eqref{eq:radial:momentum} and rearrange to attain a differential
equation for $\dR{v_R}$.
\begin{equation}
  \label{eq:regularization:equation}
  \left(\frac{2\gamma}{\gamma + 1} - \frac{v_R^2}{c_s^2}\right) \frac{1}{v_R} \dR{v_R} = \frac{R(\Omega_K^2 - \Omega^2)}{c_s^2} - \frac{2\gamma}{\gamma+1}\left(\frac{1}{R} - \frac{1}{\Omega_K}\dR{\Omega_K}\right) - \frac{\gamma-1}{\gamma+1} \frac{f\alpha R}{v_R} \dR{\Omega}.
\end{equation}

Then, to force $\dR{v_R}$ to stay well behaved, we impose that both
sides of equation \eqref{eq:regularization:equation} are zero and
attain
\begin{eqnarray}
  \label{eq:regularity:1}
  v_R^2 &=& \frac{2\gamma}{\gamma + 1} c_s^2\\
  \label{eq:regularity:2}
  (\Omega_K^2 - \Omega^2)R &=& c_s^2 \frac{2\gamma}{\gamma+1}\left(\frac{1}{R} - \frac{1}{\Omega_K}\dR{\Omega_K}\right) + c_s^2 \frac{\gamma-1}{\gamma+1} \frac{f\alpha R}{v_R} \dR{\Omega}.
\end{eqnarray}

% \subsubsection{Boundary Data at the Sonic Point}
% \label{subsubsec:initial:data:at:the:sonic:point}

% The regularization conditions give us two pieces of initial data at
% the sonic point. But we need three pieces of initial data. To get the
% final piece of initial data, we evaluate equation
% \eqref{eq:integrated:angular:momentum} at the outer boundary to
% calculate the integration constant $j$:
% \begin{equation}
%   \label{eq:j:solution}
%   j = \Omega(R_{out}) R_{out}^2 + \frac{\alpha c_s^2(R_{out}) R}{v_R(R_{out})}.
% \end{equation}
% Then, since $j$ is constant, we use
% \eqref{eq:integrated:angular:momentum} as initial data at the sonic
% point. However
% 
\section{Numerical Approach}
\label{sec:numerical:approach}

As we discussed in section \ref{subsec:boundary:conditions}, we use
the shooting method to solve the boundary value problem
\cite{NumericalRecipes,Heath}. The shooting method requires two
algorithms. The first algorithm is an ODE solver that solves an
initial value problem. The second algorithm is a root-finder, which
which takes the ODE solver and cleverly guesses values of $R_s$ so
that we find the correct value more quickly than if we searched by
brute force.

\subsection{Runge-Kutta Methods}
\label{subsec:runge:kutta}

Given a guess at $R_s$ and initial data, we solve the ODE system using
a ``Runge-Kutta'' algorithm. Before we define Runge-Kutta, let's first
describe a simpler, similar, method. The definition of a derivative is
\begin{equation}
  \label{eq:derivative:definition}
  \ddR\myvec{y}(R) = \lim_{h\to 0}\left[\frac{\myvec{y}(R+h) - \myvec{y}(R)}{h}\right].
\end{equation}
Or, alternatively, if $h$ is sufficiently small,
\begin{equation}
  \label{eq:forward:euler}
  \myvec{y}(R+h) = \myvec{y}(R) + h \dR{\myvec{y}}(R).
\end{equation}
If we know $\myvec{y}(R_0)$ and $\ddR\myvec{y}(R_0)$, then we can use
equation \eqref{eq:forward:euler} to solve for $\myvec{y}(R+h)$. Then,
let $R_1 = R_0 + h$ and use equation \eqref{eq:forward:euler} to solve
for $\myvec{y}(R_1 + h)$. In this way, we can solve for $\myvec{y}(R)$
for all $R > R_0$. This method is called the ``forward Euler'' method
\cite{Heath}.

Runge-Kutta methods are more sophisticated. One can use a Taylor
series expansion to define a more accurate iterative scheme that
relies on higher-order derivatives, not just first
derivatives. However, since we only have first derivative information,
we simulate higher-order derivatives by evaluating the first
derivative at a number of different values of $R$. Then, of course,
the higher-order derivatives are finite differences of these
evaluations \cite{NumericalRecipes,Heath}.

We use a fourth-order Runge-Kutta method---which means the method
effectively incorporates the first four derivatives of a
function---with adaptive step sizes: the 4(5) Runge-Kutta-Fehlberg
method \cite{Fehlberg}. We use our own implementation, which can be
found here: \cite{RKF45}.

\subsection{Integrating Inwards}
\label{subsec:integrating:inwards}

Unfortunately, the Runge-Kutta method only integrates from small $R$
to large $R$. Since we are integrating inwards, not outwards, we need
to generate an analogous ODE system that goes from $R_{out}$ to $R_s$
and not the other way around. To do this, we define some new
variables. Let
\begin{eqnarray}
  \label{eq:def:tau}
  \tau &=& \frac{R-R_{out}}{R_s - R_{out}}\\
  &=& \frac{R-10^5 R_s}{R_s(1 - 10^5)},
\end{eqnarray}
so that 
\begin{eqnarray}
  \label{eq:r:of:tau}
  R(\tau) &=& R_{out} + (R_s - R_{out}) \tau\\
  &=& R_s\left[10^5 + (1 - 10^5)\tau\right].
\end{eqnarray}
$R(\tau=0)=R_{out}$ and $R(\tau=1)=R_s$. And let $\myvec{z}$ such that
\begin{equation}
  \label{eq:def:z}
  \ddR\myvec{z} = - \myvec{O}(z,\tau),
\end{equation}
where $\myvec{O}$ is the vector-valued function in equation
\eqref{eq:y:prime:equation}. Now, to integrate inwards from $R_{out}$
we solve the system in equation \eqref{eq:def:z} by integrating
upwards from $\tau=0$. Then, for a given $R$, we can find $\tau(R)$ and
plug it into $\myvec{z}$. The result should be $\myvec{y}(R)$, where
$\myvec{y}$ is the vector of variables we care about from equation
\eqref{eq:y:definition}.

\subsection{Root Finding}
\label{subsec:root:finding}

For root finding, we use the bisection algorithm
\cite{Heath}. First, we test whether the left-hand sides of equations
\eqref{eq:regularity:1} and \eqref{eq:regularity:2} are positive or
negative. Assume without loss of generality that they are positive. If
they're negative, then multiply the equations by $-1$.

We bound $R_s$ between a maximum value and a minimum value. Call them
$R_{max}$ and $R_{min}.$ Then we guess that $R_s$ is exactly between
these two values and integrate inward from $R_{out}$ to our guessed
$R_s$, call it $R_g$. If the left-hand sides of equations
\eqref{eq:regularity:1} and \eqref{eq:regularity:2} are positive, we
chose too large of a value for $R_s$. We let $R_g$ be the new
$R_{max}$ and repeat. If they're negative, we chose too small of a value
for $R_s$. We let $R_g$ be the new $R_{min}$ and repeat. Eventually we
will converge on the correct value.

\subsection{Testing The Implementation}
\label{subsec:implementation:testing}

To actually solve the system derived in section \ref{sec:the:model},
we implement the shooting method with a 4(5) Runge-Kutta-Fehlberg
integrator as the initial-value solver and with the bisection method
as the root-finding algorithm \cite{NumericalRecipes,Heath}. We
implement both tools in C++ from scratch. We chose C++ because, even
though it contains modern programming tools such as object-oriented
and functional paradigms, it is extremely fast, often performing
within a factor of 2 more CPU cycles than C
\cite{BenchmarksGame}.\footnote{It is important to take numerical
  benchmarks with a grain of salt. Optimization and the skill of the
  programmer are substantially larger factors in code performance
  \cite{WrongData}.} 

We implemented and tested the Runge-Kutta initial-value solver
separately. The code is open-sourced and the implementation and unit
tests can be found online at \cite{RKF45}. In addition, we tested the
algorithm by using it as the time-stepping component in a simple
(1+1)-dimensional wave equation solver, using the method of lines (for
more information, see e.g., \cite{Heath}). This test can also be found
online at \cite{WaveEquation}. We tested the implementation of the
root solver and the shooting method less intensely since these tools
are less sophisticated.\footnote{We tested the root solver on a simple
  polynomial and found it to behave as expected.} The implementation
of the entire algorithm and a few simple tests can be found at
\cite{AccretionShooting}. To convince you that the algorithm
functioned correctly, we now briefly describe some of the tests
performed and their significance.

\subsubsection{High-Order One Point Integration}
\label{subsubsec:one-point}

A fourth-order Taylor series method should be able to solve the
initial value problem
\begin{equation}
  \label{eq:fourth:order:IVP}
  \begin{cases}
  \frac{d^4 y}{d t^4} = C_4\ \forall\ t\in\R\\
  \frac{d^3 y}{d t^3}\eval_{t=0} = C_3\\
  \frac{d^2 y}{d t^2}\eval_{t=0} = C_2\\
  \frac{d y}{dt}\eval_{t=0} = C_1\\
  y(t=0) = C_0,
  \end{cases}\text{ where }C_i=\text{constant}\ \forall\ i\in\{0,1,2,3,4\}
\end{equation}
in a single computer time step. This is because an equation like
\eqref{eq:fourth:order:IVP} contains only fourth-order
information. Similarly, a fifth-order Taylor series method should be
able to solve a sufficiently simple initial value problem in a single
time step. 

To test our method, We performed this test on the fourth-order
integration and the fifth-order error estimation pieces of our
Runge-Kutta-Fehlberg method and found discrepancy on the order of
machine epsilon, the round-off error inherent in the discretization of
the real number line. Figure \ref{fig:polynomial:integration} shows
the results of one such integration test. If the reader is so
inclined, they can find this test in the unit test driver in
\cite{RKF45}.

\begin{figure}[h!t!b!]
  \begin{center}
    \leavevmode
    \includegraphics[width=12cm]{../figures/rkf45_for_fourth_order_polynomial.png}
    \caption[4(5) Runge-Kutta Method for a Fourth-Order
    Polynomial]{4(5) Runge-Kutta Method for a Fourth-Order
      Polynomial. We demonstrate that our Runge-Kutta method is indeed
      a fourth-order Taylor series method by integrating a simple
      fourth-order initial value problem in a single computer time
      step. Each red dot is an integration starting at $t=0$ to
      $t=x$. In each case, the result is a perfect $x^4$.}
    \label{fig:polynomial:integration}
  \end{center}
\end{figure}

\subsubsection{A Simple Stability Test}
\label{subsubsec:simple:stability}

As a simple test of stability, we use the Runge-Kutta initial-value
solver to solve the $2^{nd}$-order ODE system
\begin{equation}
  \label{eq:SHO}
  \frac{d^2 y(t)}{d t^2} = -y(t)\text{ with } \begin{cases}y(0)=0\\y'(0)=1\end{cases}.
\end{equation}
The solution to this problem is naturally 
\begin{equation}
  \label{eq:SHO:solution}
  y(t) = \sin(t).
\end{equation}

\begin{figure}[htb!]
  \centering
  \includegraphics[width=14cm]{../figures/rk45_with_1-1000th_relative_and_absolute_error.png}
  \caption[A simple stability test]{A high-order integration of
    equation \eqref{eq:SHO} compared to the analytic solution. Over
    time, even a high-accuracy solution slowly diverges because the
    higher-order terms of the Taylor series are never zero.}
  \label{fig:SHO:1}
\end{figure}

Figure \ref{fig:SHO:1} shows a typical high-accuracy integration of
this equation and compares it to the expected value. An oscillatory
solution is a good test of the stability of a Taylor-series method
because the Taylor series always has non-zero higher-order terms that
are not captured. And thus error builds up from truncating the Taylor
series. Moreover, we roughly estimate the buildup of error by summing
up the difference between a fourth-order and a fifth-order
approximation at each time step. If the actual measured error is less
than this, we're in very good shape. To investigate this, we require a
very low error tolerance of the integrator and measure its deviation
based on local truncation error. Figure \ref{fig:SHO:2} shows the
results.

\begin{figure}[htb!]
  \begin{center}
    \leavevmode
    \includegraphics[width=14cm]{../figures/rkf45_truncation_error_low_accuracy.png}
    \caption[A simple stability test]{A low-order comparison of total
      error buildup as expected by the local truncation error compared
      to the actual measured deviation from the analytic solution. The
      total error is well within tolerance.}
  \end{center}
  \label{fig:SHO:2}
\end{figure}

\section{Results}
\label{sec:results}

In the attempted simulations we assume the following parameters:
\begin{eqnarray}
  \label{eq:choices}
  M = 10 M_{\odot} & a_* = 0 \nonumber\\
  f = 3.5\times 10^{-5} & \alpha = 0.1\\
  \gamma = 1.5&\dot{M} = -(0.11)\frac{4\pi G M m_p}{(1-f)c\sigma_T},\nonumber  
\end{eqnarray}
where $\dot{M}$ is $11\%$ the Eddington accretion rate, $\sigma_T$ is
the Thomson scattering cross-section, and $m_p$ is the mass of a
proton.

Unfortunately, the shooting method never correctly determined the
sonic radius. In figures \ref{fig:speed:condition} and
\ref{fig:angular:condition}, we plot the regularization conditions
\eqref{eq:regularity:1} and \eqref{eq:regularity:2} at the guessed
``sonic point'' after guessing a sonic radius and integrating
inwards. The conditions both diverge. Condition
\eqref{fig:speed:condition} diverges to $-\infty$ while condition
\eqref{eq:regularity:2} diverges to $+\infty$.

\begin{figure}[h!t!b!]
  \begin{center}
    \leavevmode
    \includegraphics[width=12cm]{../figures/speed_condition_plot.png}
    \caption[Constraints]{Evaluation of the regularization condition
      \eqref{eq:regularity:1} at the guessed ``sonic point'' after
      integrating inwards from the outer boundary. The condition
      should evaluate to zero at the sonic point. However, it instead
      diverges to $-\infty$. The units on the $y$-axis are $m^2/s^2$.}
    \label{fig:speed:condition}
  \end{center}
\end{figure}

\begin{figure}[h!t!b!]
  \begin{center}
    \leavevmode
    \includegraphics[width=12cm]{../figures/angular_constraint.png}
    \caption[Constraints]{Evaluation of the regularization condition
      \eqref{eq:regularity:2} at the guessed ``sonic point'' after
      integrating inwards from the outer boundary. The condition
      should evaluate to zero at the sonic point. However, it instead
      diverges to $+\infty$. The units on the $y$-axis are $m/s^2$.}
    \label{fig:angular:condition}
  \end{center}
\end{figure}

Surprisingly, these values line up perfectly with the values of the
constraint equations evaluated on the self-similar solution described
in section \ref{subsubsec:self-similar} at the guessed sonic
radius. This indicates that the solution is not changing very much as
we integrate from the outer boundary, and indeed a direct examination
of the integration data confirms this. This static behavior is in
direct contradiction with Shafee et al., who predict significant
changes in every variable.

We suspect that the difficulty arises from integrating inwards from
the outer boundary as opposed to outwards from the sonic
point. Although it is not at all obvious, when the sonic point
regularity conditions are true, equations \eqref{eq:ODE:system:dv},
\eqref{eq:ODE:system:dc2s}, and \eqref{eq:ODE:system:domega} become
indeterminate and read $0/0$. In practice, sufficiently close to the
sonic point, this might cause the numerical algorithm to evaluate all
derivatives as zero and prevent the system from evolving as it
should. And until the sonic point is reached, the self-similar
solution is just that, self-similar. It doesn't change much anyway.

It might be possible to salvage the shooing method by using equation
\eqref{eq:integrated:angular:momentum} at the outer boundary to
calculate the integration constant $j$:
\begin{equation}
  \label{eq:j:solution}
  j = \Omega(R_{out}) R_{out}^2 + \frac{\alpha c_s^2(R_{out}) R}{v_R(R_{out})}.
\end{equation}
Then, since $j$ is constant, use
\eqref{eq:integrated:angular:momentum} and the regularity conditions
\eqref{eq:regularity:1} and \eqref{eq:regularity:2} to solve for
initial data at the sonic point and integrate outwards instead of
inwards. Unfortunately, the resulting system of algebraic equations is
highly-nonlinear and must be solved numerically. In dimensions greater
than one, root-finding methods for nonlinear systems of equations are
very slow and unreliable \cite{NumericalRecipes,Heath}, and using the
shooting method to integrate outwards implies running a root-finding
algorithm for every single guess of $R_s$. This seemed very
inefficient, which was why we didn't attempt it in the first place.

Shafee et al. didn't use the shooting method. Instead, they used a
class of methods called ``relaxation methods,'' \cite{Shafee08}
wherein one iteratively chooses a discretization for the ODE system
that matches the boundary conditions \cite{NumericalRecipes}. In
retrospect, it is likely that Shafee and collaborators predicted that
the shooting method would be too naive. In a future attempt to
generate a physically meaningful solution, we recommend the use of a
dedicated boundary-value solver algorithm like a relaxation method.

\section{Summary and Discussion}
\label{sec:conclusion}

In summary, we were able to reproduce the analytic calculations of
Shafee et al. \cite{Shafee08} starting from Shakura and Sunyaev in
1973 \cite{ShakuraSunyaev73} and working our way forwards through the
history of axisymmetric accretion disks with effective viscosity
\cite{PBK81,MuchotrzebPaczynski82,Kato88a,Abramowicz88,PophamNarayan91,NarayanPopham93,ChenTaam93,Narayan97,Chen97,AfshordiPaczynski03,NarayanYi94}. For
the fluid dynamics equations, We use the $\alpha$ prescription of
Shakura and Sunyaev \cite{ShakuraSunyaev73}, the pseudo-Newtonian
gravitational potentials of Mukopadhyay \cite{Mukhopadhyay02} and
Paczynski and Wiita \cite{PW80}. For the boundary data, we regularize
the ODE system at the sonic point according to Shafee et al.'s method
\cite{Shafee08} and we impose self-similarity conditions at the outer
boundary, according to Spruit et al. \cite{Spruit87} and Narayan et
al, \cite{Narayan97}.

We attempt to solve the resulting first-order ODE system by using the
shooting method and integrating inwards from the outer boundary using
a Runge-Kutta integrator. We are able to demonstrate the accuracy and
stability of our numerical toolbox. However, the integration fails and
remains mostly static and self-similar. We believe the problem stems
from how the regularization conditions at the sonic point interact
with the Runge-Kutta integrator. Therefore, a future attempt must
either solve the regularization conditions to attain initial data
integrate outwards or use a dedicated boundary-value problem solving
tool such as a relaxation methods algorithm.

Although we failed to reproduce Shafee et al.'s results, we feel that
this work is a valuable exploration of the methods in and history of
accretion disk physics and offers valuable insight into which
numerical approaches are appropriate for this sort of problem.

\section{Acknowledgments}
\label{sec:acknowledgments}

I would like to thank Professor Erik Schnetter for his advice on
numerical methods and for understanding my time constraints. I'd also
like to thank Scott VanBommel and Katherine White for many helpful
discussions. Finally, I'd like to thank Professor Niayesh Afshordi for
a fun, educational semester, and for assigning this challenging and
interesting project.

\appendix
\section{Demonstration that $H\approx c_s/\Omega_K$}
\label{app:H:relation}

Assume an axisymmetric thin disk orbiting around a central Newtonian
compact object. Let the object have mass $M$ and let the radial
distance from the central object be $R$. Let the height above the
plane of rotation be $H$. Let the vector pointing out from the central
object be $\vec{r}$. This vector can be decomposed into a radially
outward component and a vertical component as follows
\begin{equation}
  \label{eq:r:decomposition}
  \vec{r} = R\hat{R} + z\vec{z},
\end{equation}
where $\hat{R}$ is the radial direction outward and $\hat{z}$ is the
direction out of the plane of rotation.

The vertical component of the gravitational field due to the star is
\begin{equation}
  \label{eq:vertical_field}
  g_z = -g_* \sin(\theta) = -\frac{G M}{r^2}\sin(\theta),
\end{equation}
where $\theta$ is the angle between $\hat{r}$ and the orbital plane
and $g_*$ is the full gravitational field due to the
star\cite{ThortonMarion}. For a thin disk confined to roughly a single
orbital radius ($r\approx R$ and $\theta << 1$),
\begin{equation}
  \label{eq:gz:new}
  g_z = -\frac{G M}{R^3} z.
\end{equation}
But if the disk is in a circular orbit (and it should circularize),
then it has Keplerian angular velocity \cite{ThortonMarion}
\begin{equation}
  \label{eq:keplerian:angular:velocity:appendix}
  \Omega_K = -\sqrt{\frac{GM}{R^3}}.
\end{equation}
So, 
\begin{equation}
  \label{eq:gz:final}
  g_z = -\Omega_K^2 z.
\end{equation}

If we assume hydrostatic equilibrium, we have Euler's equation
\begin{eqnarray}
  \label{eq:euler}
  \frac{d\vec{u}}{dt} &=& \frac{1}{\rho}\myvec{\nabla} P + \myvec{\nabla}\Phi = 0,\\
  \Rightarrow \frac{1}{\rho}{\myvec{\nabla} P} &=& -\myvec{\nabla}\Phi = g
\end{eqnarray}
where $\vec{u}$ is the velocity vector field, $\rho$ is the density of
the disk, $P$ is the pressure of the disk, and $\Phi$ is the
gravitational potential \cite{Melia,Thompson}. If we confine ourselves
to the z direction, this gives us that
\begin{equation}
  \label{eq:vertical:hydrostatic}
  \frac{1}{\rho}\frac{dp}{dz} = -\Omega_K^2 z.
\end{equation}
Now, recall that the speed of sound $c_s$ is defined by \eqref{eq:p:rho:relation} as
\begin{equation}
  \label{eq:def:cs}
  P = c_s^2 \rho.
\end{equation}
Then
\begin{eqnarray}
  \label{eq:hydrostatic:lhs}
  \frac{1}{\rho}\frac{dP}{dz} &=& \frac{c_s^2}{\rho}\frac{d\rho}{dz}.
\end{eqnarray}
Then we have 
\begin{eqnarray}
  \label{eq:hydrostatic:solving}
  \frac{c_s^2}{\rho}\frac{d\rho}{dz} &=& -\Omega_K^2 z\\
  \Rightarrow c_s^2 \int_{\rho_0}^{\rho_H} \frac{1}{\rho'}d\rho' &=& -\Omega_K^2 \int_0^H z dz\\
  \Rightarrow c_s^2\ln\left(\frac{\rho_0}{\rho_H}\right) &=& \frac{1}{2}\Omega_K^2 H^2\\
  \Rightarrow \frac{c_s}{H} &=& \Omega_K \left[2\ln\left(\frac{\rho_0}{\rho_H}\right)\right]^{-1/2},
\end{eqnarray}
where $\rho_H$ is the density at height $H$ and $\rho_0$ is the
density at the orbital plane. Of course, $\ln(x)$ grows incredibly
slowly, and $\sqrt{\ln(x)}$ grows equally slowly. So the $\ln$ term is
of order 1 and, for a rough approximation, we can safely neglect it:
\begin{equation}
  \label{eq:final:cs}
  H \approx \frac{c_s}{\Omega_K}.
\end{equation}

%Bibliography
\newpage
\bibliography{bh_accretion}
\bibliographystyle{unsrt}

\end{document}